\documentclass[a4paper,14pt]{scrreprt}

\usepackage[left=2cm,right=2cm,top=2cm,bottom=3cm,bindingoffset=0cm]{geometry}
\usepackage{polyglossia}
\usepackage[nointegrals]{wasysym}
\usepackage{fontspec}

\setmainfont{Liberation Serif}
\setsansfont{Liberation Sans}
\setmonofont{Liberation Mono}
\newfontfamily\cyrillicfont{Liberation Sans}
\defaultfontfeatures{Scale=MatchLowercase, Mapping=tex-text}

\setmainlanguage{russian}
\setotherlanguage{english}
\setkeys{russian}{babelshorthands=true}

\usepackage{amsmath}
\usepackage{amssymb}
\usepackage{graphicx}


% Title Page
\title{Домашняя работа № 4}
\author{Потоцкая Анастасия\\Б8303а}

\newcommand{\lmd}{\lambda}
\newcommand{\un}[1]{\underline{#1}}

\usepackage{tikz}


\DeclareMathOperator{\fact}{fact}
\DeclareMathOperator{\false}{false}
\DeclareMathOperator{\true}{true}
\DeclareMathOperator{\pair}{pair}
\DeclareMathOperator{\mif}{if}
\DeclareMathOperator{\mnot}{not}
\DeclareMathOperator{\miszero}{iszero?}
\DeclareMathOperator{\mmult}{mult}
\DeclareMathOperator{\mpre}{pre}


\begin{document}
\maketitle
\section*{№15}
Построить решение для уравнения для $ \fact$ и проверить его.
	$$ \fact \equiv Y ( \lmd gn. \mif (\miszero n)\un{1} (\mmult n (g(\mpre n)))) $$
	$$ \text{по свойству оператора  } Y : YF = F(YF) $$
	$$ ( \lmd gn. \mif (\miszero n)\un{1} (\mmult n (g(\mpre n)))) (Y ( \lmd gn. \mif (\miszero n)\un{1} (\mmult n (g(\mpre n))))) \to $$
	$$ \lmd n. \mif (\miszero n)\un{1} (\mmult n Y ( \lmd gn. \mif (\miszero n)\un{1} (\mmult n (g(\mpre n)))) ) \to $$
	$$ \lmd n. \mif (\miszero n)\un{1} (\mmult n (\fact(\mpre n)) )$$

\section*{№16}
Доказать, что терм $ \Theta = A A $ , где $ A=\lmd xy. y(xxy)$ есть комбинатор неподвижной точки.
$$ \Theta F \equiv AAF \to (\lmd xy.y(xxy))(\lmd xy.y(xxy))F \to $$
$$ \to F((\lmd xy.y(xxy))(\lmd xy.y(xxy))F ) \to F(AAF) \equiv F(\Theta F) $$
Доказано, что терм $ \Theta $ есть комбинатор неподвижной точки.

\section*{№17}
Определены ли термы:
\begin{enumerate}
	\item $Y \equiv \lmd f.(\lmd x.f(xx))(\lmd x.f(xx))$ \\
	$Y \to \lmd f.f((\lmd x.f(xx))(\lmd x.f(xx)))$ - HNF. Терм определен.
	
	\item $Y \mnot \equiv (\lmd f.(\lmd x.f(xx))(\lmd x.f(xx))) \mnot$
		$$ Y\mnot \to (\lmd x.\mnot(xx))(\lmd x.\mnot(xx)) \to $$
		$$ \to  \mnot((\lmd x.\mnot(xx))(\lmd x.\mnot(xx))) \to $$
		$$ \to  (\lmd x.\mif x \false\true)((\lmd x.\mnot(xx))(\lmd x.\mnot(xx))) \to $$
		$$ \to  \mif ((\lmd x.\mnot(xx))(\lmd x.\mnot(xx))) \false\true\to $$
		$$ \to ((\lmd x.\mnot(xx))(\lmd x.\mnot(xx))) \false\true \to $$
		$$ \to   Y\mnot \false \true $$
	После normal order reduction пришли к тому же уравнению. Значит $Y \mnot $ не имеет HNF. Терм не определен.

	\item $K \equiv\ lmd xy.x$ \\ Терм в HNF.Терм определен.
	
	\item $YI \equiv (\lmd f.(\lmd x.f(xx))(\lmd x.f(xx)))(\lmd x.x)$
		$$ YI \to (\lmd x.(\lmd x.x)(xx))(\lmd x.(\lmd x.x)(xx)) \to $$
		$$ \to  (\lmd x.(xx)) (\lmd x.(xx)) \equiv \Omega $$
	$\Omega$ не имеет HNF. Терм не определен.
	
	\item $x\Omega \equiv \lmd.x\Omega$ - HNF. Терм определен.
	
	\item $YK\equiv(\lmd f.(\lmd x.f(xx))(\lmd x.f(xx)))(\lmd xy.x)$
	$$ YK\to(\lmd x.(\lmd xy.x)(\un{x} \un{x} )) \un{(\lmd x.(\lmd xy.x)(xx))} \to $$
	$$ \to (\lmd xy.x) ((\lmd x.(\lmd xy.x)(xx))(\lmd x.(\lmd xy.x)(xx))) \to $$
	$$ \to \lmd y. ((\lmd x.(\lmd xy.x)(xx))(\lmd x.(\lmd xy.x)(xx))) \to \lmd y.YK$$
	После normal order reduction пришли к тому же уравнению. Значит $YK$ не имеет HNF. Терм не определен.
	
	\item $\un{n} \equiv \lmd fx.f^n x \equiv \lmd fx.f(f(\dots f(fx)))$ - HNF. Терм определен.

\end{enumerate}
\end{document}
