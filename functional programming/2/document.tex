\documentclass[a4paper,14pt]{scrreprt}

\usepackage[left=2cm,right=2cm,top=2cm,bottom=3cm,bindingoffset=0cm]{geometry}
\usepackage{polyglossia}
\usepackage[nointegrals]{wasysym}
\usepackage{fontspec}

\setmainfont{Liberation Serif}
\setsansfont{Liberation Sans}
\setmonofont{Liberation Mono}
\newfontfamily\cyrillicfont{Liberation Sans}
\defaultfontfeatures{Scale=MatchLowercase, Mapping=tex-text}

\setmainlanguage{russian}
\setotherlanguage{english}
\setkeys{russian}{babelshorthands=true}

\usepackage{amsmath}
\usepackage{amssymb}
\usepackage{graphicx}


% Title Page
\title{Домашняя работа № 2}
\author{Потоцкая Анастасия\\Б8303а}

\newcommand{\lmd}{\lambda}
\newcommand{\un}{\underline}

\DeclareMathOperator{\msucc}{succ}

\begin{document}
\maketitle

\section*{№6}
Добавим аксиому $\lmd xy.x=\lmd xy.y$. Доказать, что тогда любые 2 терма равны.

Доказательство: 
	$$ \forall M,N: \ (\lmd xy.x)MN \to_\beta M; \; (\lmd xy.y)MN \to_\beta N \Rightarrow M = N$$

\section*{№7}
Задача: показать верность определения $\text{mult}\equiv\lmd mnfx.m(nf)x$

Решение:
\begin{align*}
	\text{mult}\ \un{m} \ \un{n} &\to \lmd fx.\un{m}(\un{n}f)x \to \lmd fx.(\lmd fx.f^m x)(\un{n}f)x\to\\
	&\to\lmd fx.(\un{n}f)^m x\to \lmd fx.(f^n)^m \to \lmd fx. f^{nm} x= \un{m} \ \un{n}
\end{align*}

\section*{№8}
Задача: показать верность определения $\text{exp}\equiv\lmd mnfx.nmfx=m^n$

Решение:
\begin{align*}
\text{exp}\ \un{m}\ \un{n} &\to \lmd fx.\un{n} \ \un{m}fx\to \lmd fx.(\lmd fx.f^nx)\un{m}fx\to\\
&\to \lmd fx.\un{m}^nfx \to \lmd fx.f^{m^n}x=\un{m}^n
\end{align*}

\section*{№9}
Задача: показать, что $ \un{m} \msucc \un{n} \equiv  \un{m} + \un{n}$.
Решение:
$$ \msucc \un{n} \to \lmd fx.f(\un{n}fx) \to \lmd fx.f(f^nx) \equiv \lmd fx.f^{n+1}x \equiv \un{n+1}$$
$$ \un{m} \msucc \un{n} \to \msucc^{m} \un{n} \to \msucc^{m-1}(\msucc \un{n}) \to $$
$$ \to \msucc^{m-1} \un{n + 1} \to \dots \to \un{n+m} $$

\end{document}
