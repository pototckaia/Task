\documentclass[a4paper,14pt]{scrreprt}

\usepackage[left=2cm,right=2cm,top=2cm,bottom=3cm,bindingoffset=0cm]{geometry}
\usepackage{polyglossia}
\usepackage[nointegrals]{wasysym}
\usepackage{fontspec}

\setmainfont{Liberation Serif}
\setsansfont{Liberation Sans}
\setmonofont{Liberation Mono}
\newfontfamily\cyrillicfont{Liberation Sans}
\defaultfontfeatures{Scale=MatchLowercase, Mapping=tex-text}

\setmainlanguage{russian}
\setotherlanguage{english}
\setkeys{russian}{babelshorthands=true}

\usepackage{amsmath}
\usepackage{amssymb}
\usepackage{graphicx}


% Title Page
\title{Домашняя работа № 3}
\author{Потоцкая Анастасия\\Б8303а}

\newcommand{\lmd}{\lambda}

\DeclareMathOperator{\false}{false}
\DeclareMathOperator{\true}{true}
\DeclareMathOperator{\pair}{pair}
\DeclareMathOperator{\mnull}{null?}
\DeclareMathOperator{\mcons}{cons}
\DeclareMathOperator{\mhead}{head}
\DeclareMathOperator{\mtail}{tail}
\DeclareMathOperator{\mfirst}{first}
\DeclareMathOperator{\msecond}{second}


\begin{document}
\maketitle

\section*{№10}
Показать, что   
$$ \mnull (\mcons M N)= \false $$
$$\mhead  (\mcons M N)= M$$
$$\mtail(\mcons M N)=N$$

\begin{enumerate}
\item 
$$ \mnull (\mcons M N) \equiv (\lmd z.\mfirst z)((\lmd xy. \pair \false (\pair xy)) M N) \to $$
$$ \to \mfirst (\pair \false (\pair M N))) \to \false $$
\item 
$$\mhead  (\mcons M N) \equiv (\lmd z. \mfirst(\msecond z)) (\pair \false (\pair MN)) \to  $$
$$ \to \mfirst(\msecond (\pair \false (\pair MN)) ) \to \mfirst (\pair MN) \to M $$ 
\item 
$$ \mtail(\mcons M N) \equiv (\lmd x. \msecond (\msecond z)) (\pair \false (\pair MN)) \to $$
$$ \to \msecond (\msecond (\pair \false (\pair MN))) \to \msecond (\pair MN) \to N $$
\end{enumerate}

\section*{№11}
Описать натуральные числа использую списки. Ввести на них операцию сложения и умножения. \\
Зададим натуральные числа ввиде.
$$ \underline{0} = \text{nil} $$
$$ \underline{1} = \text{cons} \; x \; \text{nil} $$
$$ \underline{2} = \text{cons} \; x (\text{cons} \; x \;  \text{nil}) $$
$$ \underline{n} = \underbrace{\text{cons} (\dots \text{cons} \; x \; \text{nil})}_{n}  $$ 
Тогда операции сложения и умножения примут вид
$$ \text{add} \equiv \text{append} $$
$$ \text{mult} \equiv \lmd NZ. \text{if} \; (\text{null?} \; Z) \underline{0}\; (\text{append}\; N (\text{mult}\; N (\text{tail}\; Z))   ) $$ 

\section*{№12}
Показать, что функции prefn и pre удовлетворяют cпецификации. Рассмотрим функцию 
$$\text{prefn} \equiv \lmd fp. \text{pair} \; (f(\text{first} \; p))(\text{first} \; p) $$
Покажем, что 
$$ \underline{n+1} \; (\text{prefn} \; f) (\text{pair} \; xx) = \text{pair} \; (f^{n+1}x) (f^{n}x) $$
Доказательство
$$ \underline{n+1} \; (\text{prefn} \; f) (\text{pair} \; xx) = 
(\text{prefn} \; f)^{n+1} (\text{pair} \; xx) = 
(\text{prefn} \; f)^{n} \; \text{prefn} \; f \; (\text{pair} \; xx) =$$
$$ = (\text{prefn} \; f)^{n} \; \text{pair} (fx)x = 
(\text{prefn} \; f)^{n-1} \text{prefn} \; f \; \text{pair} \; (fx)x = $$
$$ = (\text{prefn} \; f)^{n-1} \; \text{pair} \; (f^2 x) fx =  \text{pair} \; (f^{n+1}x) (f^{n}x) $$
Покажем теперь, что 
$$ \text{pre} \; \underline{n+1} = \lmd fx. \; \text{second} \;(\underline{n+1} \; (\text{prefn} \; f)(\text{pair} \;xx)) = $$
$$ = \lmd fx. \text{second} \; (\text{pair} \; (f^{n+1}x) (f^nx)) 
=  \lmd fx.f^nx = \underline{n}$$ 
$$ \text{pre} \; \underline{0} = \lmd fx. \text{second} \; (\underline{0} \; (\text{prefn} \; f)(\text{pair} \; xx)) =  
\lmd fx.\text{second} \; (\text{pair} \; xx) = \lmd fx.x = \underline{0} $$ 

\section*{№13}
Показать
$$ \text{sub} \; \underline{m} \; \underline{n} \rightarrow \underline{m - n}$$
Доказательство
$$ \text{sub} \; \underline{m} \; \underline{n} \rightarrow \underline{n} \; \text{pre} \; \underline{m} 
 \rightarrow \text{pre}^{n} \; \underline{m} \rightarrow \text{pre}^{n-1} \text{pre} \; \underline{m}  $$
$$ \rightarrow \text{pre}^{n-1} \; \underline{m-1} \rightarrow \underline{m - n}  $$

\section*{№14}
Показать, что для функции 
$$ \text{ack}
 \equiv \lmd m. m(\lmd fn. nf(f\underline{1})) \; \text{suc} $$
выполненый свойства
\begin{enumerate}
\item $$ \text{ack} \; \underline{m + 1}\: \underline{0} = \text{ack} \; \underline{m}\: \underline{1} $$
\item $$ \text{ack} \; \underline{m + 1}\: \underline{n + 1} = \text{ack} \; \underline{m} (\text{ack} \;  \underline{m + 1} \: \underline{n}) $$
\end{enumerate}
Рассмотрим выражение 
$$ \text{ack} \; \underline{m + 1}\: \underline{n} = \underline{m + 1} (\lmd fn. nf(f\underline{1})) \; \text{suc} \; \underline{n} $$ 
$$ \underline{m + 1} = \lmd fx.f^{m + 1}x = \lmd fx. ff^{m}x = \lmd fx.f(\underline{m} fx) $$
$$ \text{ack} \; \underline{m + 1}\: \underline{n} = (\lmd fn. nf(f\underline{1}))\; (\underline{m} \; (\lmd fn. nf(f\underline{1})) \; \text{suc}) \underline{n} = $$
$$ = (\lmd fn. nf(f\underline{1}))\; (\text{ack} \; \underline{m}) \; \underline{n} = 
\underline{n} \; (\text{ack} \; \underline{m}) (\text{ack} \; \underline{m} \; \underline{1}) $$
Рассмотрим первое свойствой
$$ \text{ack} \; \underline{m+1} \; \underline{0} = \underline{0} \; (\text{ack} \; \underline{m}) (\text{ack} \; \underline{m} \; \underline{1}) = 
 \text{ack} \; \underline{m} \; \underline{1} $$
Рассмотрим второе свойство
$$ \text{ack} \; \underline{m + 1}\: \underline{n + 1}  
= \underline{n+1} (\text{ack} \; \underline{m}) (\text{ack} \; \underline{m} \; \underline{1}) = $$
$$ = \text{ack} \; \underline{m} \; (\underline{n} \; (\text{ack} \; \underline{m}) (\text{ack} \; \underline{m} \; \underline{1}))
=  \text{ack} \; \underline{m} (\text{ack} \;  \underline{m + 1} \: \underline{n}) $$ 

\end{document}
